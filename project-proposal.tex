\documentclass{project-proposal}

\begin{document}

%=====================================================================================%
% Settings to Apply to Entire Document
%=====================================================================================%
\setmaintextfont
\makepage

%=====================================================================================%
% Header
%=====================================================================================%
\makeheader
{Lab Integration of Multi-​Client Digital Laser Lock Box for Trapped Ion Experiments} % Title of project
{Semester Thesis} % Type of eligible project
{Spring 2024} % Type of eligible project

%=====================================================================================%
% Project Background
%=====================================================================================%
\begin{section}[assets/images/blood.png]{5.37cm}{Background}
 In 2022, our students developed a novel digital lock box, named BLOOD (Bichannel
 Lockbox On One Device), the successor to the decade-old EVIL. At its core, BLOOD is a
 PID controller designed to stabilize laser frequency by 'locking' it to a set value.
 Featuring a Red Pitaya FPGA board and a custom-designed PCB, BLOOD maintains backward
 compatibility with existing EVIL setups and our DEVIL client/server software. A major
 improvement compared to its predecessor is the capability to digitally control analog
 gain and offset parameters. The first prototype of BLOOD has shown promising results,
 successfully locking lasers in the lab. Nonetheless, further development is necessary
 to fully unlock its potential.
\end{section}

%=====================================================================================%
% Proposal
%=====================================================================================%
\begin{section}{}{Proposal}
 We propose a semester project with the following main goals and expected outcomes:

 \begin{itemize}
     \item {Continued development of BLOOD to ensure its readiness for regular lab use.}
     \item {Adding software support for the second channel of the device.}
     \item {Resolving existing issues to improve reliability and optimize performance.}
 \end{itemize}

 The project will contribute significantly to our group's research and
 give you a great learning experience.
\end{section}

%=====================================================================================%
% Recommended Skills & Learning Opportunities
%=====================================================================================%
\begin{sectiontwocolumns}
    {Recommended Skills}
    {
        Desirable (but not mandatory) skills are:

        \begin{itemize}
            \item {Proficiency in \textbf{C/C++} and \textbf{Python} for software
                  development.}
            \item {Some experience with hardware design, \textbf{FPGA} development, and
                  embedded systems.}
            \item {Familiarity with version control systems, particularly \textbf{git}.}
            \item {Basic understanding of \textbf{Linux} operating systems.}
        \end{itemize}
    }
    {Learning Opportunities}
    {
        Self-development opportunities include:

        \begin{itemize}
            \item {Gaining hands-on experience in collaborative, team-based
                  software development.}
            \item {Improving your hardware design skills, with a focus on PID
                  implementation in FPGA.}
            \item {Learning how to find and fix bugs, and make systems work better.}
            \item {Getting familiar with Linux operating systems, including
                  command line usage.}
        \end{itemize}
    }
\end{sectiontwocolumns}

%=====================================================================================%
% Timeline
%=====================================================================================%
\begin{sectiontimeline}{Timeline}

    The project aims for completion in one semester, starting February 2024:

    \timelineentry
    {Weeks 1-2}
    {Kick-off}
    {
        Initial meeting. Get familiar with BLOOD and Python-based DEVIL Client software
        (GUI and API). Use the example Jupyter Notebooks as tutorial. Fix minor bugs
        and resolve issues on GitLab. Document changes and release the updated version
        of DEVIL Client.
    }

    \timelineentryspacer

    \timelineentry
    {Weeks 3-7}
    {Server}
    {
        Begin development on C++-based Server code to enable second channel of BLOOD.
        Use the server code for EVIL as example. Build the software, upload it to FPGA
        and debug using VS Code. Document changes and release the updated version of
        DEVIL Server.
    }

    \timelineentryspacer

    \timelineentry
    {Weeks 8-11}
    {Hardware}
    {
        Upgrade PyRPL FPGA project to Vivado 2023.2, resolve any timing violations.
        Implement a PID output limiter for preventing integrator wind-up. Generate the
        bitstream and test it on the FPGA. Document changes and release the updated
        version of PyRPL FPGA bitstream.
    }

    \newpage

    \timelineentry
    {Week 12}
    {Linux OS}
    {
        Update the Buildroot submodule to the latest release and integrate the new
        versions of DEVIL Server and PyRPL FPGA packages that include your changes.
        Implement auto-generated unique hostnames for Red Pitaya. Compile and boot the
        new OS. Test it with remote access using Linux commands (ssh, systemctl).
        Document changes and release the updated version of the Red Pitaya OS.
    }

    \timelineentryspacer

    \timelineentry
    {Weeks 13-14}
    {Finish}
    {
        Test the new features of BLOOD in the lab and try to lock a laser with it.
        Prepare the final project report, detailing the development process, challenges,
        solutions and future work. Update example Jupyter Notebooks if necessary.
    }

    The project's duration, estimated at 300-400 hours, is subject to variation based
    on the specific departmental requirements (Electrical Engineering, Physics, etc.).
    This timeline presents an extensive list of tasks, serving as a best-case scenario.
    It's understood that, given the project's scope, completing every task may not be
    feasible, and prioritizing key components is acceptable.

\end{sectiontimeline}

%=====================================================================================%
% Contact
%=====================================================================================%
\begin{section}[assets/images/tiqi.png]{3cm}{Contact}

 \textbf{Lab}:\newline
 Trapped Ion Quantum Information Group\newline
 \url{www.tiqi.ethz.ch} \newline


 \textbf{Location}:\newline
 Hönggerberg Campus, HPF (E Floor)\newline

 \textbf{Supervisors}:
 \begin{itemize}
     \item {Bahadır Dönmez (doenmezb@phys.ethz.ch)}
     \item {Martin Stadler (mastadle@phys.ethz.ch)}
 \end{itemize}

\end{section}

\end{document}


