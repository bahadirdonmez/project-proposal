\documentclass{project-proposal}

\begin{document}

%=====================================================================================%
% Settings to Apply to Entire Document
%=====================================================================================%
\setmaintextfont
\makepage

%=====================================================================================%
% Header
%=====================================================================================%
\makeheader
{Lab Integration of Multi-​Client Digital Laser Lock Box for Trapped Ion Experiments} % Title of project
{Semester Thesis} % Type of eligible project
{Spring 2024} % Type of eligible project

%=====================================================================================%
% Project Background
%=====================================================================================%
\begin{section}[assets/images/blood.png]{Background}
 In 2022, our students developed a novel digital lock box, named
 \textbf{BLOOD (Bichannel Lockbox On One Device)}. This system is planned to be the
 successor to the \textbf{EVIL} lock box, which has been in use for a
 decade. BLOOD, featuring a Red Pitaya FPGA board and a custom-designed PCB,
 maintains backward compatibility with existing EVIL setups and our DEVIL client/server
 software. A major improvement in BLOOD compared to its predecessor is the
 capability to digitally control analog gain and offset parameters. The first prototype
 of BLOOD has shown promising results, successfully stabilizing laser frequencies in
 the lab. Nonetheless, further development is necessary to fully unlock its potential.
\end{section}

%=====================================================================================%
% Proposal
%=====================================================================================%
\begin{section}{Proposal}
 We propose a semester project with the following main goals and expected outcomes:

 \begin{itemize}
     \item {Continued development of BLOOD to ensure its readiness for regular lab use.}
     \item {Adding support for the second channel of the device.}
     \item {Resolving existing issues to improve reliability and optimize performance.}
 \end{itemize}

 The project will contribute significantly to our group's research and
 give you a great learning experience.
\end{section}

%=====================================================================================%
% Recommended Skills & Learning Opportunities
%=====================================================================================%
\begin{sectiontwocolumns}
    {Recommended Skills}
    {
        Desirable (but not mandatory) skills are:

        \begin{itemize}
            \item {Proficiency in \textbf{C/C++} and \textbf{Python} for software
                  development.}
            \item {Some experience with hardware design, \textbf{FPGA} development, and
                  embedded systems.}
            \item {Familiarity with version control systems, particularly \textbf{git}.}
            \item {Basic understanding of \textbf{Linux} operating systems.}
        \end{itemize}
    }
    {Learning Opportunities}
    {
        Self-development opportunities include:

        \begin{itemize}
            \item {Improving your programming and hardware design skills.}
            \item {Gaining hands-on experience in collaborative, team-based
                  software development.}
            \item {Learning how to find and fix bugs, and make systems work better.}
            \item {Getting familiar with Linux operating systems, including
                  command line usage.}
        \end{itemize}
    }
\end{sectiontwocolumns}

%=====================================================================================%
% Timeline
%=====================================================================================%
\begin{sectiontimeline}{Timeline}
    The project, estimated at 300-400 hours, aims for completion in one semester,
    starting February 2024:

    \timelineentry
    {Weeks 1-2}
    {Kick-off}
    {
        Initial meeting. Get familiar with BLOOD and Python-based DEVIL Client software
        (GUI and API). Use the example Jupyter Notebooks as tutorial. Fix minor bugs
        and resolve issues on GitLab. Document changes and release the updated version
        of DEVIL Client.
    }

    \timelineentryspacer

    \timelineentry
    {Week 3}
    {Linux OS}
    {
        Upgrade Buildroot to latest version and implement auto-generated unique
        hostnames for Red Pitaya. Compile and boot the new OS. Test with remote
        access using Linux commands (ssh, systemctl). Document changes and release the
        updated version of Red Pitaya OS.
    }

    \timelineentryspacer

    \timelineentry
    {Weeks 4-8}
    {Server}
    {
        Begin development on C++-based Server code to enable second channel of BLOOD.
        Use the server code for EVIL as example. Build the software, upload it to FPGA
        and debug using VS Code. Document changes and release the updated version of
        DEVIL Server.
    }

    \newpage

    \timelineentry
    {Weeks 9-12}
    {Hardware}
    {
        Upgrade PyRPL FPGA project to Vivado 2023.2, resolve any timing violations.
        Implement a configurable PID output limiter for preventing integrator wind-up.
        Generate the bitstream, upload it to FPGA and test it. Document changes and
        release the updated version of PyRPL FPGA bitstream.
    }

    \timelineentryspacer

    \timelineentry
    {Weeks 13-14}
    {Finish}
    {
        Test the new features of BLOOD in the lab and try to lock a laser with it.
        Prepare the final project report, detailing the development process, challenges,
        solutions and future work. Update example Jupyter Notebooks if necessary.
    }
\end{sectiontimeline}

%=====================================================================================%
% Contact
%=====================================================================================%
\begin{section}[assets/images/tiqi.png]{Contact}

 \textbf{Lab}:\newline
 Trapped Ion Quantum Information Group\newline
 www.tiqi.ethz.ch \newline

 \textbf{Location}:\newline
 Hönggerberg Campus, HPF (E Floor)\newline

 \textbf{Supervisors}:
 \begin{itemize}
     \item {Bahadır Dönmez (doenmezb@phys.ethz.ch)}
     \item {Martin Stadler (mastadle@phys.ethz.ch)}
 \end{itemize}

\end{section}

\end{document}


